\chapter{Final remarks: our contribution and some open questions}
\label{chap:conclusion}

May the principle be used to represent the landscape evolution
process in time and space using \aclp{DEM}?
\bigskip

Would a multi-objective framework be suitable to take 
advantage of the existing knowledge and to assess the trade-offs 
among different simplified version of the \enquote{optimality} 
criterion in order to find the proper expression for 3D modeling?
\bigskip

The previous two questions were written in the introduction as the
goal of this thesis. Now, after having commented the results, are
we able to answer?

\section{Our contribution to the field of study}
This thesis contributes to the field studying landscape and river
evolution as follows:
\begin{itemize}
  \item as for modeling purposes, it is the first attempt that
  provides a model to reproduce the 3D features of landscapes and
  river networks based on multiple criteria optimization;
  \item as for sistem understanding purposes, it deepened the
  knowledge about the different formulations of least action
  principle and the way they affect the features of river networks
  developing on landscapes.
\end{itemize}

\section{Proposed improvements and open questions}
Given the large amount of topics touched by this thesis, some
relevant issues are proposed to future development.