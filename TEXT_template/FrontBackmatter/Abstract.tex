%*******************************************************
% Abstract
%*******************************************************
%\renewcommand{\abstractname}{Abstract}
\addcontentsline{toc}{chapter}{\abstractname}

\pdfbookmark[1]{Abstract}{Abstract}
\begingroup
\let\clearpage\relax
\let\cleardoublepage\relax
\let\cleardoublepage\relax

\chapter*{Abstract}
In this thesis, we propose a novel Reinforcement Learning (\acs{RL}) algorithm consisting in a stochastic variance-reduced version of policy gradient for solving Markov Decision Processes (\acs{MDP}s).\newline
Stochastic variance-reduced gradient (\acs{SVRG}) methods have proven to be very successful in supervised learning. However, their adaptation to policy gradient is not straightforward and needs to account for I) a non-concave objective function; II) approximations in the full gradient computation; and III) a non-stationary sampling process. The result is \acs{SVRPG}, a stochastic variance reduced policy gradient algorithm that leverages on importance weights to preserve the unbiasedness of the gradient estimate. Under standard assumptions on the \acs{MDP}, we provide convergence guarantees for \acs{SVRPG} with a convergence rate that is linear under increasing batch sizes. Finally, we suggest practical variants of \acs{SVRPG}, and we empirically evaluate them on continuous \acs{MDP}s.

\vfill
\newpage
\pdfbookmark[1]{Sommario}{Sommario}
\chapter*{Sommario}
In questa tesi proponiamo un nuovo algoritmo nell'ambito del Reinforcement Learning (\acs{RL}). Questo algoritmo consiste  nella versione di stochastic variance-reduced del gradiente della politica per la risoluzione dei processi decisionali di Markov (\acs{MDP}s).\newline
Il metodo di riduzione della varianza del gradiente, chiamato Stochastic variance-reduced gradient (\acs{SVRG}), ha avuto molto successo nell'ambito dell'apprendimento supervisionato. La sua adattazione nel contesto del gradiente della politica non è banale e necessita di alcune considerazioni: I) la funzione obiettivo non è concava; II) il full gradient deve essere necessariamente approssimato;  III) il processo di campionamento non è stazionario. Il risultato è \acs{SVRPG}, un algoritmo di riduzione della varianza del gradiente della politica che sfrutta gli importance weights per preservare la correttezza della stima del gradiente. Date le classiche assunzioni del \acs{MDP}, abbiamo fornito garanzie di convergenza per \acs{SVRPG} con un tasso di convergenza che è lineare al crescere della dimensione del batch. Infine abbiamo implementato una variante pratica di \acs{SVRPG}. Abbiamo valutato questa variante impiricamente  su dei \acs{MDP} continui.

\endgroup