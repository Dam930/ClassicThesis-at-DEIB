%*****************************************************************
% Breve riassunto in italiano della tesi da cui si capisca tutto
% ****************************************************************
\newcommand{\estrattoname}{Estratto}
\addcontentsline{toc}{chapter}{\estrattoname}

\pdfbookmark[1]{Estratto}{Estratto}
\begingroup
\let\clearpage\relax
\let\cleardoublepage\relax
\let\cleardoublepage\relax

\chapter*{Estratto}
\begin{otherlanguage}{italian}
In molti settori, come ad esempio nel campo dell'automazione, della ricerca operativa o della robotica, esistono compiti complessi che non possono essere risolti tramite formule matematiche gi\'a pronte all'uso: la soluzione deve essere appresa sfruttando l'esperienza che un generico individuo ha del problema specifico.\newline
 L'apprendimento per rinforzo ha come obiettivo quello di fornire ad un agente artificiale tecniche per scegliere in maniera automatica il comportamento migliore da utilizzare per eseguire efficacemente un certo compito di controllo. Nello specifico l'agente non ha una conoscenza completa del problema e non necessita di un modello per rappresentarne le dinamiche interne poich\'e \'e in grado di interagire e modificare l'ambiente attraverso un certo numero di azioni e di ricevere da esso segnali di rinforzo che sono indici della bont\'a delle sue scelte.
L'agente deve decidere quale azione sia meglio utilizzare in ogni possibile situazione, che da ora in poi verr\'a chiamato stato, per massimizzare il valore dei rinforzi accumulati, cio\'e quello che \'e chiamato ritorno totale.
La politica dell'agente \'e quell'insieme di parametri che permettono all'agente, dato un particolare stato, di scegliere, da una specifica distribuzione statistica, una certa azione.
In particolare, nei controlli robotici, il numero di azioni possibili, come il numero degli stati, pu\'o essere illimitato. 
In questo caso vengono utilizzati degli algoritmi che vanno a migliorare direttamente la politica senza creare un modello del problema. 
Questo approccio \'e il pi\'u efficente ed ha avuto buoni risultati in problemi complessi, come ampiamente dimostrato in letteratura. I pi\'u noti di questa famiglia di algoritmi sono REINFORCE o G(PO)MDP.
Questi metodi si basano sull'aggiornamento della politica seguendo, di passo in passo, il gradiente della funzione obiettivo.
Per stimare correttamente il gradiente si dovrebbero provare tutti i possibili casi in cui un agente si pu\'o trovare; ci\'o \'e impossibile in quanto significa provare infinite possibilit\'a.
In questa situazione si procede con la stima del gradiente a partire da un sottoinsieme dei possibili casi, andando a calcolare quello che \'e chiamato gradiente stocastico.
Questa stima del gradiente, nell'ambito dell'apprendimento per rinforzo, presenta una elevata varianza. 
Andare ad aumentare il sottoinsieme di dati della stima \'e controproducente in quanto il campionamento dei dati \'e molto oneroso, quindi, in questa tesi, presenteremo uno studio relativo all'applicazione di una tecnica, chiamata \acf*{SVRG}, nel campo dell'apprendimento per rinforzo, la quale ci consentir\'a di ridurre la varianza dello stimatore del gradiente senza aumentare i dati di campionamento. Tale tecnica per ora trova largo impiego solo in ambito dell'apprendimento supervisionato e consiste nel calcolare il \acf*{FG} per un determinato passo della procedura di ottimizzazione, e in un numero limitato di passi successivi sfrutta questo gradiente, ormai passato, per andare a diminuire la varianza dello stimatore del gradiente utilizzato per l'ottimizzazione della funzione obiettivo. Raggiunto questo limite si calola un nuovo \acs{FG} per poi ricominciare.\newline
L'adattamento di questa tecnica nell'ambito dell'apprendimento per rinforzo prevede diverse sfide da affrontare: non \'e possibile avere un \acs{FG} esatto, ma \'e necessario considerare una stima di esso, la funzione da ottimizzare non \'e concava (questi due problemi sono gi\'a stati affrontati in letteratura separatamente nell'ambito dell'apprendimento supervisionato, ma mai in contemporanea); infine, la distribuzione dei dati di campionamento non \'e stazionaria, ma cambia al cambiare della politica. \newline
Sotto queste assunzioni siamo riusciti a dimostrare la convergenza, a regime, del nostro algoritmo ad un ottimo locale con un grado di approssimazione inevitabile dovuto all'errore nella stima del \acs{FG} e un errore dovuto alla gestione della non stazionariet\'a della distribuzione di campionamento.\newline
Abbiamo sviluppato un modello empirico del nostro algoritmo, affrontando tutte le sfide pratiche del caso. Abbiamo confrontato il nostro algoritmo con il classico metodo che utilizza il gradiente stocastico in tre problemi di controllo continuo noti in letteratura. I risultati ottenuti sono tutti soddisfacenti, in quanto la prestazione del nostro algoritmo \'e sempre migliore rispetto al metodo classico. \newline
\end{otherlanguage}

\endgroup

