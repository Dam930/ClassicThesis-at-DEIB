%*****************************************************************
% Breve riassunto in italiano della tesi da cui si capisca tutto
% ****************************************************************
\newcommand{\estrattoname}{Estratto}
\addcontentsline{toc}{chapter}{\estrattoname}

\pdfbookmark[1]{Estratto}{Estratto}
\begingroup
\let\clearpage\relax
\let\cleardoublepage\relax
\let\cleardoublepage\relax

\chapter*{Estratto}
In molti settori, come ad esempio nel campo dell'automazione, della ricerca operativa o della robotica, esistono compiti complessi che non possono essere risolti tramite formule matematiche già pronte all'uso, ma la soluzione deve essere appresa sfruttando l'esperienza che l'individuo ha del problema specifico.
 L'apprendimento per rinforzo ha come obiettivo quellodi fornire ad un agente artificiale tecniche per scegliere in maniera automatica il comportamento migliore da utilizzare per eseguire efficacemente un certo compito di controllo. Nello specifico l'agente non ha una conoscenza completa del problema e non necessita di un modello per rappresentarne le dinamiche interne poiché è in grado di interagire e modificare l'ambiente attraverso un certo numero di azioni e di ricevere da esso segnali di rinforzo che sono indici della bontà delle sue scelte.
L'agente deve decidere quale azione sia meglio utilizzare in ogni possibile situazione, o stato, per massimizzare il valore dei rinforzi accumulati, cioè quello che è chiamato ritorno totale.
La politica dell'agente è quell'insieme di parametri che permettono all'agente, dato un particolare stato, di scegliere probabilisticamente una certa azione.
In particolare, nei controlli robotici, il numero di azioni possibili, come il numero degli stati, può essere illiminato. 
In questo caso vengono utilizzati degli algoritmi che vanno a massimizzare direttamente la politica senza creare un modello del problema.

\endgroup

