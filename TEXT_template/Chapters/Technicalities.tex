\chapter{State of the art}\label{chap:art}

\vspace{-0.05in}
In this section, we provide the essential background on policy gradient methods and stochastic variance-reduced gradient methods for finite-sum optimization.
\vspace{-0.05in}

\section{Policy Gradient}\label{sec:PolicyGradient}
\vspace{-0.05in}
A Reinforcement Learning task~\citep{sutton1998reinforcement} can be modelled with a discrete-time continuous \acs{MDP} $M = \{\Sspace,\Aspace,\Transition,\Reward,\gamma,\rho\}$, where $\Sspace$ is a continuous state space; $\Aspace$ is a continuous action space; $\Transition$ is a Markovian transition model, where $\Transition(s'|s,a)$ defines the transition density from state $s$ to $s'$ under action $a$; $\Reward$ is the reward function, where $\Reward(s,a) \in [-R,R]$ is the expected reward for state-action pair $(s,a)$;
% and $R$ is the maximum absolute-value reward;
$\gamma\in[0,1)$ is the discount factor; and $\rho$ is the initial state distribution.
The agent's behaviour is modelled as a policy $\pi$, where $\pi(\cdot|s)$ is the density distribution over $\Aspace$ in state $s$.
% We consider episodic tasks, \ie tasks composed of episodes of length $H$, also called time horizon.
% In this setting, a trajectory $\tau$ is the sequence of states and actions $(s_0,a_0,s_1,a_1,\dots,s_{H-1},a_{H-1})$ observed by following some stationary policy in a given episode, where $s_0$ is always sampled from the initial state distribution $\rho$.
We consider episodic \acs{MDP}s with effective horizon $H$.\footnote{The episode duration is a random variable, but the optimal policy can reach the target state (\ie absorbing state) in at most $H$ steps. This has not to be confused with a finite horizon problem where the optimal policy is non-stationary.} In this setting, we can limit our attention to trajectories of length $H$. A trajectory $\tau$ is a sequence of states and actions $(s_0,a_0,s_1,a_1,\dots,s_{H-1},a_{H-1})$ observed by following a stationary policy, where $s_0 \sim \rho$.
We denote with $p(\tau|\pi)$ the density distribution induced by policy $\pi$ on the set $\Tspace$ of all possible trajectories (see Appendix~\ref{A:gradient_estimators} for the definition), and with $\Reward(\tau)$ the total discounted reward provided by trajectory $\tau$:
%
$\Reward(\tau) = \sum_{t=0}^{H-1}\gamma^t\Reward(s_t,a_t).$
%
Policies can be ranked based on their expected total reward: $J(\pi) = \EVV[\tau \sim p(\cdot|\pi)]{\Reward(\tau)|M}$.
Solving an \acs{MDP} $M$ means finding $\pi^* \in \argmax_{\pi} \{J(\pi)\}$.

Policy gradient methods restrict the search for the best performing policy over a class of parametrized policies $\Pi_{\vtheta}=\{\pol: \vtheta \in \realspace^d\}$, with the only constraint that $\pol$ is differentiable \wrt $\vtheta$. For sake of brevity, we will denote the performance of a parametric policy with $J(\vtheta)$ and the probability of a trajectory $\tau$ with $p(\tau|\vtheta)$ (in some occasions, $p(\tau|\vtheta)$ will be replaced by $p_{\vtheta}(\tau)$ for the sake of readability).
The search for a locally optimal policy is performed through gradient ascent, where the policy gradient
%of $J(\vtheta)$ \wrt the policy parameters 
is \citep{sutton2000policy, Peters2008reinf}:
\begin{align} \label{E:policygradient}
	\gradJ{\vtheta} = \EVV[\tau \sim p(\cdot|\vtheta)]{\score{\vtheta}{\tau}\Reward(\tau)}.
	%\int_{\Tspace}\pol(\tau)\score{\vtheta}{\tau}\Reward(\tau)\de \tau.
\end{align}
Notice that the distribution defining the gradient is induced by the current policy. This aspect introduces a nonstationarity in the sampling process. Since the underlying distribution changes over time, it is necessary to resample at each update or use weighting techniques such as importance sampling.
Here, we consider the \emph{online learning scenario}, where trajectories are sampled by interacting with the environment at each policy change. 
In this setting, stochastic gradient ascent is typically employed.
At each iteration $k >0$, a batch $\mathcal{D}_N^k = \{\tau_i\}_{i=0}^N$ of $N>0$ trajectories is collected using policy $\pi_{\vtheta_k}$.
The policy is then updated as $\vtheta_{k+1}  = \vtheta_k + \alpha\gradApp{\vtheta_k}{N}$, where $\alpha$ is a step size and $\gradApp{\vtheta}{N}$ is an estimate of Eq.~\eqref{E:policygradient} using $\mathcal{D}_N^k$. The most common policy gradient estimators (\eg REINFORCE~\citep{williams1992simple} and G(PO)MDP~\citep{baxter2001infinite}) can be expressed as follows
\begin{align} \label{E:policygradient.estimate}
	\gradApp{\vtheta}{N} = \frac{1}{N}\sum_{n=1}^{N} g(\tau_i|\vtheta), \quad \tau_i \in \mathcal{D}_N^k,
\end{align}
where $g(\tau_i|\vtheta)$ is an estimate of $\score{\vtheta}{\tau_i}\Reward(\tau_i)$.
Although the REINFORCE definition is simpler than the G(PO)MDP one, the latter is usually preferred due to its lower variance.
We refer the reader to Appendix~\ref{A:gradient_estimators} for details and a formal definition of $g$.

The main limitation of plain policy gradient is the high variance of these estimators.
The na\"ive approach of increasing the batch size is not an option in \acs{RL} due to the high cost of collecting samples, \ie by interacting with the environment.
For this reason, literature has focused on the introduction of baselines (\ie functions $b : \mathcal{S} \times \mathcal{A} \to \realspace$) aiming to reduce the variance~\citep[\eg][]{williams1992simple,Peters2008reinf,Thomas2017actionbaseline,wu2018variance}, see Appendix~\ref{A:gradient_estimators} for a formal definition of $b$.
These baselines are usually designed to minimize the variance of the gradient estimate, but even them need to be estimated from data, partially reducing their effectiveness.
On the other hand, there has been a surge of recent interest in variance reduction techniques for gradient optimization in supervised learning (\acs{SL}).
Although these techniques have been mainly derived for finite-sum problems, we will show in Section~\ref{sec:alg} how they can be used in \acs{RL}.
In particular, we will show that the proposed \acs{SVRPG} algorithm can take the best of both worlds (\ie \acs{SL} and \acs{RL}) since it can be plugged into a policy gradient estimate using baselines.
The next section has the aim to describe variance reduction techniques for finite-sum problems. In particular, we will present the \acs{SVRG} algorithm that is at the core of this work.


\vspace{-0.05in}
\section{Stochastic Variance-Reduced Gradient}\label{sec:svrg}
\vspace{-0.05in}
Finite-sum optimization is the problem of maximizing an objective function $f(\vtheta)$ which can be decomposed into the sum or average of a finite number of functions $z_i(\cdot|\vtheta)$:
\begin{align*}
        \max_{\vtheta} \left\{ f(\vtheta) = \frac{1}{N}\sum_{i=1}^{N}z_i(\vtheta)\right\}.
\end{align*}
This kind of optimization is very common in machine learning, where each $z_i$ may correspond to a data sample $x_i$ from a dataset $\mathcal{D}_N$ of size $N$ (\ie $z_i(\vtheta) = z(x_i|\vtheta)$). 
%% In this case, we adopt the following, more meaningful notation:
%% \begin{align*}
%% \max_{\vtheta} f(\vtheta) = \frac{1}{N}\sum_{i=1}^{N}g(x_i \vert \vtheta).
%% \end{align*}
A common requirement is that $z$ must be smooth and concave in $\vtheta$.\footnote{Note that we are considering a maximization problem instead of the classical minimization one.} 
Under this hypothesis, \acs{FG} ascent~\citep{cauchy1847methode} with a constant step size achieves a linear convergence rate in the number $T$ of iterations (\ie parameter updates)~\citep{nesterov2013introductory}.
However, each iteration requires $N$ gradient computations, which can be too expensive for large values of $N$. Stochastic Gradient (\acs{SG}) ascent~\citep[\eg][]{robbins1951stochastic,bottou2004large} overcomes this problem by sampling a single sample $x_i$ per iteration, but a vanishing step size is required to control the variance introduced by sampling. As a consequence, the lower per-iteration cost is paid with a worse, sub-linear convergence rate \cite{nemirovskii1983problem}.
Starting from SAG, a series of variations to \acs{SG} have been proposed to achieve a better trade-off between convergence speed and cost per iteration: \eg SAG~\citep{roux2012stochastic}, \acs{SVRG}~\cite{johnson2013accelerating}, SAGA~\cite{defazio2014saga}, and Finito~\cite{defazio2014finito}. 
The common idea is to reuse past gradient computations to reduce the variance of the current estimate.
In particular, Stochastic Variance-Reduced Gradient (\acs{SVRG}) is often preferred to other similar methods for its limited storage requirements, which is a significant advantage when deep and/or wide neural networks are employed.  

The idea of \acs{SVRG} (Algorithm~\ref{alg:svrg}) is to alternate full and stochastic gradient updates. 
Each $m = \mathcal{O}(N)$ iterations, a snapshot $\widetilde{\vtheta}$ of the current parameter is saved together with its \acs{FG} $\nabla f(\widetilde{\vtheta}) = \frac{1}{N} \sum_i \nabla z(x_i|\widetilde{\vtheta})$.
Between snapshots, the parameter is updated with $\blacktriangledown f(\vtheta)$, a gradient estimate corrected using stochastic gradient. For any $t \in \{0,\ldots,m-1\}$:
\begin{equation}\label{E:svrg.gradient.correction}
        \blacktriangledown f(\vtheta_{t}) := v_t = \nabla f(\wt{\vtheta}) + \nabla z(x | \vtheta_t) - \nabla z(x | \wt{\vtheta}),
\end{equation} 
where $x$ is sampled uniformly at random from $\mathcal{D}_N$ (\ie $x \sim \mathcal{U}(\mathcal{D}_N)$).
Note that $t=0$ corresponds to a \acs{FG} step (\ie $\blacktriangledown f(\vtheta_0) = \nabla f(\wt{\vtheta})$) since $\vtheta_0 := \wt{\vtheta}$.
The corrected gradient $\blacktriangledown f(\vtheta)$ is an unbiased estimate of $\nabla f(\vtheta)$, and it is able to control the variance introduced by sampling even with a fixed step size, achieving a linear convergence rate without resorting to a plain \acs{FG}.

More recently, some extensions of variance reduction algorithms to the non-concave objectives have been proposed~\citep[\eg][]{allen2016variance,reddi2016stochastic,reddi2016fast}. In this scenario, $f$ is typically required to be L-smooth, \ie $\norm[2]{\nabla f(\vtheta') - \nabla f(\vtheta)} \leq L\norm[2]{\vtheta'-\vtheta}$ for each $\vtheta,\vtheta'\in\realspace^n$ and for some Lipschitz constant $L$. Under this hypothesis, the convergence rate of \acs{SG} is $O(\nicefrac{1}{\sqrt{T}})$ \cite{ghadimi2013stochastic}, \ie $T=O(\nicefrac{1}{\epsilon^2})$ iterations are required to get $\norm[2]{\nabla f(\vtheta)}^2\leq\epsilon$. Again, \acs{SVRG} achieves the same rate as \acs{FG} \cite{reddi2016stochastic}, which is $O(\frac{1}{T})$ in this case \cite{nesterov2013introductory}. The only additional requirement is to select $\vtheta^*$ uniformly at random among all the $\vtheta_k$ instead of simply setting it to the final value.

\vspace{-0.05in}